\begin{abstract}
In this bachelor thesis we attempt to modify the existing face registration pipeline for the morphable face model of Prof. Thomas Vetter by using a registration algorithm developed by PD Marcel Lüthi at the University of Basel. ALTERNATIVE:
In this bachelor thesis we discuss the construction of a face registration pipeline using an algorithm based on a vector-valued gaussian process and at the same time attempting to ensure registration quality through the use of contours marking important parts of the face - referred to as line features.

The algorithm is capable of mapping any two shapes on to one another. All that is needed is a set of corresponding points on the two shapes. Different constraints to the displacement field can be applied through regularisation. 

The aim of this bachelor thesis is more specifically to apply this general algorithm for point correspondences to scanned face data, that is to implement feasible registration of face scans onto the mean face of the morphable model. 
In order to achieve this we mark important parts of the face meshs not only with point landmarks, but also structures and organs (eyebrows, eyes, ears) with lines - line features - and thereby to create further correspondences for the algorithm to perform better by. 
Instead of using sparse points of key features points of the face we mark complex features, e.g. the eyes, with contour lines - line features in order to create further correspondences 

These line features are marked by hand using bézier curves on three 2D images to the front, left and right of the 3D face. 
In order to utilize them, however, they have to be projected on to the computed mesh of the face that was recorded by a 3D scanner. These meshs have holes in the region of the eyes and the ears rendering the projected line features useless at first. This thesis first gives an overview over the morphable model and the face registration pipeline, then goes on to obtaining 3D points from the 2D line features, to explain the theory behind the general algorithm and in the main part discusses
the problems and solutions we encountered trying to optimize the algorithm for and without line features for the face registration process.

\end{abstract}
