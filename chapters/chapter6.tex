\chapter{Discussion \& Future Work}
In this thesis we discussed a new approach to 3D face registration with Gaussian Process Regression incorporating line features. As demonstrated in chapter \ref{chap:results} the quality of the registration was significantly enhanced through the use of line features. The registration pipeline yields deformed templates that provide a good matching of the feature regions with those of the respective targets. This is largely due to the attempted mapping of feature regions between template and target using the line features. For
transforming the 2D line features we were obliged to use an approximative approach where the different appearances of the holes in the respective scans carry the risk of influencing the results. The ideal case would of course be to find a parametric 3D representation of the line features that could be plugged directly in to the registration process. The constructed registration pipeline is in its current state a good basis to further build upon and refine the registration algorithm which already yields fits of the template mesh resulting in profound likeness with the respective targets. However, there is still room for improvement, particularly concerning the expressiveness and the exact modelling of the feature regions of the obtained registration results. The results lack in accuracy, because of the necessitated use of a robust estimator as the loss function which is due to the overhang of the template mesh and the face scans. This overhang also causes artifacts despite the fact that a robust estimator is deployed. An improvement of the registration scheme therefore would be to first detect the vertices in the template mesh corresponding with the holes in the target mesh by registering the template with it. With the outlier regions detected, the Mean Squared Error could be used to achieve very accurate registration results without the appearance of artifacts. The proposed approaches will be subject of future work on this topic.\\  
In conclusion, the presented approach to 3D face registration successfully incorporated line features to establish correspondence which lead to more expressive and accurate registration results.

\begin{comment}
Tried out regularization
mount of basis functions
Adpation of the optimization scheme. 
squared exponential covariance function, for example to make it less smooth and give it a smaller domain.
The use of vector-valued Gaussian Processes can be further adapted to 3D face registration by either changing the covariance function or tweaking the parameters of the  Of course, the expressiveness and exact modelling in the feature regions of the fits has to be enhanced as
In order to the pipeline to yield near-perfect results Gaussian Process Regression and the optimization process have to be further adapted an improved. In essence, the imperative must be to find a good balance of parameter settings in the covariance function, the parametric representation of the Gaussian Process Posterior and the loss function.

\end{comment}
