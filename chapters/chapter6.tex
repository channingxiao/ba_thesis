\chapter{Discussion \& Future Work}
In this thesis we discussed a new approach to 3D face registration with Gaussian Process Regression and incorporating line features. We constructed a registration pipeline that uses the samples from corresponding line features as additional prior information to define the sought deformation fields. We assumed an equidistant parametrization to be the best approach to gain approximately corresponding points on the line features of the template and respective targets. In order to be able to sample equidistant
points from the line features, we had to first compute an equidistant parametrization of the B\'{e}zier curves that define them. The 3D representation of the samples were obtained by projecting them on to the respective target mesh. The Gaussian Process Posterior distribution could then be conditioned on the residuals of the corresponding sample points in template and target mesh as well as residuals of the  corresponding landmarks. Using the Mean Squared Error as the loss function in the optimization procedure produced results with heavy artifacts, so we tried to minimize the effect of outlier regions using robust estimators.
As demonstrated in chapter \ref{chap:results} the quality of the registration was significantly enhanced through this process.\\
The registration pipeline yields deformed templates that provide a good matching of the feature regions with those of the respective targets. This is largely due to the attempted mapping of feature regions between template and target using the line features. For
transforming the 2D line features we were obliged to use an approximative approach where the different appearances of the holes in the respective scans carry the risk of influencing the results. We thereby assumed that the line features define exact correspondence by being marked in the same manner on every feature contour. The implemented method is not robust to line features that are slightly shifted, because we start sampling at the origin of the first curve segment.
Furthermore, if projections are off target with a greater distance than the variance of the noise model it leads to a distortion in the fitting result.
The ideal case would of course be to find a parametric 3D representation of the line features that could be plugged directly in to the registration process. A method for obtaining the representation of a whole curve in 3D to apply a depth constraint to curve points along the visual lines from the focal point of the camera model. For this to be implemented the noise model of the Gaussian Process has to be extended.\\
The constructed registration pipeline is in its current state a good basis to further build upon and refine the registration algorithm. The results are promising seeing that the deformed templates reflect the characteristics of the respective targets. However, there is still room for improvement, particularly concerning the expressiveness of parts of the obtained faces that are not defined by line features, for example the chin and cheeks. The results lack in accuracy, because of the necessitated use of a robust estimator as the loss function which is
due to certain areas of the face and head missing in either the template or the target. These missing regions still cause artifacts in some cases of registration despite the fact that a robust estimator is used. A possible improvement of the registration scheme therefore would be to first detect the vertices in the template mesh corresponding with the holes in the target mesh by registering the template with it. With the outlier regions detected, the Mean Squared Error could be used to achieve very accurate registration results without the appearance of artifacts. The proposed approaches will be subject of future work on this topic.\\  

\begin{comment}
Tried out regularization
mount of basis functions
Adpation of the optimization scheme. 
squared exponential covariance function, for example to make it less smooth and give it a smaller domain.
The use of vector-valued Gaussian Processes can be further adapted to 3D face registration by either changing the covariance function or tweaking the parameters of the  Of course, the expressiveness and exact modelling in the feature regions of the fits has to be enhanced as
In order to the pipeline to yield near-perfect results Gaussian Process Regression and the optimization process have to be further adapted an improved. In essence, the imperative must be to find a good balance of parameter settings in the covariance function, the parametric representation of the Gaussian Process Posterior and the loss function.

In conclusion, the presented approach to 3D face registration successfully incorporated line features to establish correspondence which lead to more expressive and accurate registration results.

\end{comment}

