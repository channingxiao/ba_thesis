\chapter{Introduction}

\section{Problem Statement}
So es bizzeli alles schriebe
1. Use Gaussian Processes - 2. Use Line Features => prepare for Gaussian Process Regression
In this bachelor thesis 
Implement 3D face registration using Gaussian Processes and Line Features. One part of the problem is to sample equidistant 3D points from 2D line features marked on images of a 3D face scan. These line features should then be used as an additional input to a registration algorithm which is based on Gaussian Process Regression. The aim is to build a pipeline which starts off with the raw scan data as well as the landmarks and line features. The feature points are used to register the mean face of the MM/BFM (Basel Face Model) on to/with the raw scan thereby obtaining a fully defined and textured 3D model representation of the face in 3D. Registration is the technique of aligning to objects using a transformation, in this case the registration is performed by adding displacements to every points in the mean face model.
A model is represented as vector N*d. What is a model? A vector representation of a 3D scan?
For the morphing a Posterior Shape Model is used in combination with a Gaussian Process.
Image registration is a process of aligning two images into a common coordinate system 
thus aligning.

(gaussian process + line features for accurate, reproducable registration)

\section{Review Literature}
	2. Definition of terms (morphable model, 3D face registration, Gaussian Process regression, posterior shape models)
	3. Review of literature (papers)


