\chapter{Introduction}
\begin{comment}3D Face Models have a wide variety of applications. They are used in\end{comment}
Face recognition from 2D images, 3D face reconstruction from 2D images and the determination of facial expressions in face image analysis are all applications of 3D Face Models. Registration is the first step in the process of deriving such 3D models of morphable faces from a set of facial scans. It is used to establish the required dense correspondence between the input scans via a transformation. This transformation is normally computed with the help of a set of corresponding
points and assures that the vertices defining the surface of the scans are semantically structured such that they describe the same parts of the face. \begin{comment}Correspondence betweenk two faces is established by deforming a template face on to a target face.\end{comment} Defining a reproducable and accurate registration algorithm is therefore essential to obtaining a high-quality 3D model. 
In this thesis we present a registration algorithm which tackles the problem of 3D face registration by modeling the transformation using Gaussian Process Regression (GPR) and extending the prior information used in the transformation to parametric curves - referred to as line features - which define key regions of the face.

\section{Prior Work}
Gaussian Processes for Machine Learning were comprehensively introduced in [1]
%(Gaussian Processes for Machine Learning - Carl Edward Rasmussen and Christopher K. I. Williams)
as an approach for defining a prior distribution over a space of functions. A Gaussian Process (GP) is a stochastic process, which in turn can be seen as the generalization of probability distributions to functions. Computations prove to be straightforward because a GP is modelled by a multivariate normal distribution. In the Graphics and Vision Group at the University of Basel Gaussian Processes have so far been used to describe transformation priors in Image Registration
[2] (Using Landmarks as a Deformation Prior for Hybrid Image Registration). In this context GP allows for defining a space of possible transformations of an image and further allows for incorporating point correspondences as additional information for inference of admissible deformations. The distinction in regard to other registration methods lies therein that the corresponding points are not hard-wired in the transformation, but instead modify the prior to include only the
more feasible transformations. In [3] the process of
formulating the fitting and optimization of the admissible transformations is described for image registration. GPR has further been used to infer shapes in statistical shape models [4]. In short, the reason for applying GPR is its reproducability in modelling a deformation prior for abritrary shapes.\\ 
The registration methods used so far [5] in 3D face registration are specifically designed for this context are therefore not very well reproducable. Furthermore, these registration methods have to be directly constrained on point correspondences in the optimization process, whereas with GPR only admissible transformations have to be optimized. In effect, the advantage of applying GPR to the problem of 3D face registration is that it is based on Bayesian Machine Learning Theory, allows for
the easy incorporation of a prior known correspondences and is therefore highly customizable.

\section{Overview}
In face registration the different a common parametrization has to be established for all face scans. Using a GP in 3D face registration models a space of possible prior
deformations of a facial scan for the purpose of registering it with another scan. The given a priori point correspondences - called landmarks - are used as additional information constraining the space of possible deformations. The line features, introduced above, describe complex regions of the face that have a large abundance of surface vertices, which have to be brought into dense correspondence with those of the other scans. With the help of line features we want to infer
deformations that result in a more accurate mapping of these regions and therefore better
registration results. \begin{comment}As GPR is an inference technique we can further directly derive the optimization.\end{comment}
Using GPR in the registration process results in a highly customizable algorithm, that is  dependent on the covariance of the surface points on the a priori given information. 
The following chapters cover the basics of building 3D morphable models, GPR theory and the implementation of a face registration
pipeline with the above-mentioned algorithm. The registration results shown in the last chapter give a first expression of what this method is capable of and how the incorporation of line features affects the registration outcome. 
The question of how to incorporate the line features in the registration process will be answered in the course of this thesis.

\begin{comment}
One of the major questions beforehand was how to incorporate 
One of the major problems was how to use line features in a way that they could be used by the algorithm to further constrain the space of admissible deformations.
How to incorporate information of the 2D line features into the Gaussian Process?

start with idea of how thesis will be structured. What will be covered in each subsequent chapter?
start broad, explain purpose of thesis
talk about specific things related to your niche
something unexpected
why does work matter? to help justify own work
best work in the field, what gaps of knowledge remain? Brian ambergs registration method is not set in general theory, GPR is
conclusion, but set up sense of anticipation
summarise general principles, state problem or question that needs an answer and give quick hint how following chapters will help to answer question

Registration is the method of making data from different measurements or even sources comparable by applying a transformation. It has many applications, for example in computer vision and medically imaging.
\end{comment}
