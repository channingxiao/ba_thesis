\chapter{3D Model Building}
This chapter describes how to build a generative textured 3D face model from an example set of 3D face scans. A morphable model is derived from the set of scans by transforming their shape and texture into a vector space representation. The term generative implies that new faces can be generated by calculating linear combinations of the set of examples. 

\section{3D Morphable Model}
The 3D Morphable Model (3DMM) published by Blanz and Vetter in 1999 (bib) is a multidimensional function for modelling textured faces derived from a a large set of $m$ 3D face scans. A vector space can be constructed from the available data set where each face is represented by a shape-vector $S \in \mathbb{R}^{3n}$ that contains all three coordinates of its n vertices. The texture-vector $T \in \mathbb{T}^{3n}$ contains the corresponding RGB values. New shapes and textures can now be computed
with a linear model parametrized by barycentric shape $\vec\alpha \in \mathbb{R}^{m}$ and texture coefficients $\vec\beta \in \mathbb{R}^{m}$.\\
However, the goal of such a 3D face model is not to construct arbitrary faces, but plausible faces. This is achieved by estimating two multivariate normal distributions for the coefficients in $\vec\alpha$ and $\vec\beta$.
By observing the likelihood of the coefficients it is now possible to find out how likely the appearance of a corresponding face is.
The multivariate normal distributions are constructed from the average shapes $\overline{S} \in \mathbb{R}^{3N}$ and textures $\overline{T} \in \mathbb{R}^{3N}$ of the datasets and the covariance matrices $K_{S}$ and $K_{T}$, which are defined over the differences between each example and the average in both shape and texture.
The covariance matrices are then used to perform a Principal Component Analysis which defines a basis transformation to an orthogonal coordinate system the axis of which are the eigenvectors of the respective covariance matrices.

\begin{equation}
\label{eq:MM}
\mathcal{S}(\vec\alpha)=\overline{S}+S\vec\alpha, \quad \mathcal{T}(\vec\beta)=\overline{T}+T\vec\beta)
\end{equation}

In \eqref{eq:MM} the $N=m$ principal eigenvectors of $K_{S}$ and $K_{T}$ respectively are assembled column-wise in S and T and scaled in a way such that the prior distribution over the shape and texture parameters is given by a multivariate normal distribution with unit covariance (Amberg).

\begin{equation}
    \mathbb{p}(\vec\alpha, \vec\beta) = \mathcal{N}(\vec\alpha\vert|\vect{0}, \mathbb{I})\mathcal{N}(\vec\beta\vert|\vect{0}, \mathbb{I})
\end{equation}

\section{Achieving Correspondence through Registration}
In order for a 3D Morphable Model to generate plausible faces we have to make sure that all faces in the example set are equally parametrized by triangulated meshs. For this reason the meshs first have to be brought into correspondence, meaning that all faces share the same mesh triangulation, respectively the vertices at the same semantical position, i.e. the corner of the eye, have a similar vertex number. Correspondence is achieved/accomplished through the process of registration. 
The training data used for learning a 3D Morphable Models consists solely of registered examples of the 3D shape and texture of faces.

Correspondence at salient features (corners of the mouth) blabla
Difficult to define correspondences on in-between points
Registration algorithm chooses a smooth deformation of reference/template mathcing surface and feature points
\paragraph{Registration}
parametrizing one shape in terms of another shape, such that semantically corresponding points are mapped onto each other. This parametrization can also be seen as a deformation of the reference shape onto the target shape.
Registration is usually achieved by 
Registration algorithm

``Having constructed a parametric face model that is able to generate almost any face, the correspondence problem turns into a mathematical optimization problem
New faces, images or 3D face scancs can be registered by minimizing the difference between the new face and its reconstruction by the face model function.``
The key problem is to compute a dense point-to-point correspondence between the vertices of the faces''

New method of establishing registration

\section{Prerequisite Data}
Describe data and scanner given + Camera model?

In the next chapter we will elaborate on the approach of using Gaussian Processes to solving the problem 3D face registration.

\nopagebreak



